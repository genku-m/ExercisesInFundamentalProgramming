\documentclass[11pt,a4j]{jarticle}
\usepackage{graphicx}
\title{コンピュータリテラシレポート#09}
\author{2120029 政野玄空}
\date{2021年6月18日}
\begin{document}
\maketitle


\section{演習3}
LaTeXをつかって自分の自己紹介の文章を作成する。
\begin{itemize}
\item 箇条書き、引用を使う。箇条書きの中に箇条書きを含める。
\item 脚注を使う。できればその中で使えない機能がなにか調べる。
\end{itemize}
\section{レポートの本文}
\subsection{自己紹介}

私は1993年に兵庫県篠山市\footnote{現在は市名が変更され丹波篠山市になった。}に生まれました。
中学では野球をやっていましたが、高校からギターを始めて現在でも趣味で続けています。
最近はベースギターもよく弾きます。
今は東京都に居を構え、ソフトウェア開発の仕事に従事しています。

\subsection{趣味}
\begin{itemize}
\item ゲーム
\item ジャズの演奏
\item 睡眠\footnote{寝不足気味なので早く課題を終わらせて寝たい気持ち。}
\begin{itemize}
\item 8時間は寝たい
\item 最短4時間
\item Zzzz
\begin{itemize}
\item 寝てます\footnote{起こさないでください。\footnote{起きません。\footnote{起きません。}}}
\end{itemize}
\end{itemize}
\end{itemize}



\section{考察}
箇条書きをとりあえず3つまで入れ子にできることがわかった。
脚注の中に脚注をぶちこむと採番だけされた。見えていないだけなのかはよくわからない。さらに入れ子にすると何も見えない。

\section{アンケート}

\subsection{Q1:普段どのくらい文書を作成していますか。またそのときに使うソフトや心がけていることなどを教えてください。}
普段は仕事ではKibelaというソフトウェアを使い文章を作成しています。社内の色々な人が見るのでわかりやすい文章になるよう心がけています。plantumlとかも使ったりします。

\subsection{Q2:LaTeXのようなマークアップ型の文書作成系についてどのように感じましたか。}
便利だと思いました。

\subsection{Q3:リフレクション (今回の課題で分かったこと)・感想・要望をどうぞ。}
ググったら仕事で使っているkibelaでもlatex記法で色々できることがわかったので試してみようと思った。

\end{document}