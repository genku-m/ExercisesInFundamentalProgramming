\documentclass[11pt,a4j]{jarticle}
\usepackage[dvipdfmx]{graphicx}
\title{コンピュータリテラシレポート#10}
\author{2120029 政野玄空}
\date{6月26日}
\begin{document}
\maketitle

\section{演習1-c}
PBMで好きな画像を作る

\section{作成した画像}

適当なドット絵が作れるソフトでドット絵を書く

作成した画像は図\ref{fig1}のようなものになった。
\begin{figure}[htbp]
\begin{center}
\includegraphics[width=10cm]{flower.eps}

\caption{図です}\label{fig1}
\end{center}
\end{figure}

\section{考察}
最近買ったゲームで出てくるきれいな花を書いた。
ピクセル数が増えてくると人力ではとても大変なのでコードを書いて自動生成できるようにしたほうがいいと思った。
画像を取り込んで分割して色を抽出し並べ直す作業をコードに落とし込めると複雑な写真をpbmで吐き出したりするのができそうだと思った。
\section{アンケート}

\subsection*{Q1. ピクセルグラフィクスとベクターグラフィクスについてどれくらい知っ ていましたか。新たに知って面白かったことは何ですか。}
初めて触った。昔のゲームっぽかった。
\subsection*{Q2. LaTeXでの図表の扱いや参照機能についてどう思いましたか。}
便利だと思った。
\subsection*{Q3. リフレクション(今回の課題で分かったこと)・感想・要望をどうぞ。}
ちゃんと準備をしていなかったので時間がかかってしまった。
\end{document}
