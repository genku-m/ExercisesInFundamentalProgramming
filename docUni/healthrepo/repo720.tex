\documentclass[12pt,a4j]{jarticle}
\begin{document}
\title{健康実践論レポート}
\author{学籍番号2120029, 氏名 政野玄空}
\date{7月20日}
\maketitle

\section{テーマ}
7/6に行った筋肉トレーニングの目的、方法、成果と反省点について
\section{本文}
7/6に5つの筋肉トレーニングを行ったのでそれぞれ目的、方法、成果と反省点を記述する。
\subsection{ベンチプレス}
まずベンチプレスを行った。主に3つ、
\begin{itemize}
\item 大胸筋
\item 三角筋
\item 上腕三頭筋
\end{itemize}
を鍛えるトレーニングである。大胸筋は腕立て伏せでも鍛えられるので自分を腕で持ち上げたり、なにかを押したりするときに使う。三角筋は腕や肩を動かすすべての動作に使われる。上腕三頭筋は肘を伸ばすときに使われる。具体的にはものをおろしたりするときである。
方法はまず台に横になり、バーベルを持ち上げ、シャフトが胸につくまで下ろすを繰り返す。このとき手首がかえらないように注意するのと、バーベルをバランスよく持てるようにシャフトの位置に注意する。回数は20kgのものを10回を2Set行った。
成果は上腕三頭筋あたりが疲れた感じがした。筋肉痛や疲労は感じられるほどなかった。
反省点としては重さが足りなかったため、成果があまり出なかった点である。
\subsection{スクワット}
次にスクワットを行った。主に
\begin{itemize}
\item 大腿四頭筋
\item 大でん筋
\end{itemize}
が鍛えられる。大腿四頭筋は太ももあたりの4つの筋肉の総称で全身の筋肉の中で最も大きい部分であり、トレーニングの効果が高い部位である。大でん筋はお尻の筋肉で歩いたりするときに使われる。
方法はバーベルを肩に背負う、腰を低く落とす、元の体勢に戻る、を繰り返すものである。注意点は足先が膝より前に出ないように後ろに座るように注意してやること。回数は20kgのものを10回を2Set行った。
成果は繰り返しているうちに太ももに疲れを感じた。筋肉痛が感じられるほどではなかった。
反省点としては重さが足りなかったため、成果があまり出なかった点、もしくは回数が足りなかった点である。
\subsection{ローイング}
次にローイングというものを行った。主に鍛えられるのは背中にある筋肉全体で、具体的に1つ上げると広背筋である。
方法は足を少し曲げてマシンの前に座る。マシンをそのまま引っ張るを繰り返す。注意点は背筋を曲げないようにすること。回数は38kgを10回行った。
成果は少し背中の筋肉が回数を重ねるにつれ疲れていっているように感じた。筋肉痛が感じられる程ではなかった。
反省点としては重さが足りなかったことと気を抜いていると姿勢がわるくなっていたことである。
\subsection{ラットプルダウン}
次にラットプルダウンというものを行った。こちらも背中にある筋肉、主に広背筋が鍛えられるトレーニングである。
方法はマシンに座る。マシンを首の後ろあたりまで引っ張る。これを繰り返す。こちらも姿勢が悪くならないように注意する。回数は32kgを10回行った。
成果は少し腕の筋肉、三頭筋のあたりが疲れた。筋肉痛が感じられる程ではなかった。
反省点としては重さが足りなかったこととである。
\subsection{ダンベルカール}
最後にダンベルを用いたトレーニングを行った。主に上腕二頭筋が鍛えられるトレーニングでものを持ち上げるときに使う。
方法は手を伸ばした状態で片手にダンベルを持つ。まっすぐ持ち上げる。これを繰り返す。腕がまっすぐ上がるように注意してやる。7kgのものを左右で10回行った。
成果は今回の中で一番あった。後日上腕二頭筋に筋肉痛がみられた。
反省点としては特になく良いトレーニング担ったと思う。
\section{まとめ}
以上が授業のなかで行ったトレーニングである。今回は重さが足りないと感じることが多かったので自分にあった重さを見つけて効率的にトレーニングできるように意識したい。
\end{document}