\documentclass[12pt,a4j]{jarticle}
\usepackage{graphicx}
\usepackage{url}
\begin{document}
\title{コンピュータリテラシレポート#11}
\author{2120029, 政野玄空\\
グループのメンバー\\
飯田陸斗2120002\\
倉ヶ崎玲央 2120012\\
政野玄空 2120029\\
山村ひろし 2120034\\
小山祐 2120008\\
}
\date{7月2日}
\maketitle


\section{討議の準備としてしたこと}
紙の本には価値があるか? ネットで詳しく新鮮で動画なども含んだ情報を見た方がずっとよくないか?という題だったのでまず、自身の生活の中で、紙の本と電子書籍どちらを読むことが多いかを確認した。
結果、テレワークの影響もあって家にいることが多いため、ほとんどの場合紙の本を読んでいることがわかったので紙の書籍のほうがよいとして意見する想定で準備を始めた。
最近まで引っ越したばかりで本を床に雑多に並べていたが、本棚を購入し見える所に並べたことによってこの本のこういうところにこのような情報がある、この本でわからないところがあったなどの読んだときに記憶が引っ張り出しやすいのではないかという仮説を立てた。
記憶に残りやすいという意見もあると思うが、一般の人の記憶力はしれているので紙の本でも電子書籍でもかわりないと感じたので、記憶力よりは、思い出せることが重要だと感じた。
またネットの情報に対して利点を示すため、陳腐化しにくい情報、例えば数学や物理学など陳腐化しにくい技術、知識などの歴史を重ねた洗練性、信頼性があると主張しようと考えた。
\section{討議の内容}
まずそれぞれが意見を述べることになった。
倉ケ崎さんが電子書籍の進化は目に見えてあるが、紙の本はここ数年変化がないという主張をした。一方で勉強するときは紙の本は有用であり状況によって価値があるとした。
次に飯田さんがネットに通じていて見れる情報がおおくまた動画や画像でも情報を得られ紙の本にこだわる必要はないと主張した。
次に山村さんが紙の本の価値が下がってきていると主張した。ほとんどの状況で紙の本でなくても同様の情報が得られることが多く、こだわる必要がないと主張した。
次に小山さんが紙の本には価値があると主張した。出版社を通して発行することに価値があること、紙の本は理解力が向上するという研究があるということ、著作権に関することで本は貸し借りが違法にならないことを主張した。
また紙の本だれかが読んだ本やとても古くコレクションとして価値があると付け加えた。
最後に自分は準備した内容を主張した。

話は討論に進んでいき山村さんは紙が物理的にあることで場所を消費してしまっている、また物理的にあることで様々な制約があり、破損したりする可能性がありデメリットが有ると主張した。
破損に関しては情報媒体(HDD,SSD)は100年単位で情報を残すには適切なメンテナンスが必要であり制約はあるとした。
それに関連してデジタル情報にたいして、リテラシーが必要という話になった。その情報が改ざんされていないかなどの判断は難易度が高いと意見が出たが、自分はそれに対して情報の信頼性は今後の技術の向上、例えばブロックチェーン等改ざん性に強いものを利用できれば情報としての信頼性を保てると意見した。
また売る側の意見として古本等は利益にならないので電子書籍の方が売る側の人々には価値が大きいとした。

\section{結論}
結論として紙の本には物理的に存在することで価値を生み出す部分が多く、全く価値がなくなったとは言い切れないのではという話になった。
紙の本だとデメリットに感じることを電子書籍やネットで解決していく、両方のメリットを享受できるようにつかうのが最も効果が高く紙の本の価値も引き出しやすいと自分の中で結論づけた。
\section{考察}
同じ立場でもさまざまな視点からの意見が出たり対抗意見の方でもある意見に対してそれはこういう理由で違うのではという話が出てきて、立場を明確に分けて議論をするということは徹底できていなかった。
一方でそれぞれが様々な視点で考えるきっかけになり、その場で出版側の立場などについて考察できたのはとてもよかった。

\section{参考文献}

\begin{thebibliography}{99}

    \bibitem{howlifeunfolds} \TeX 「紙の本を読むことで理解力向上」: \url{https://www.howlifeunfolds.com/learning-education/7-scientific-benefits-reading-printed-books}

    \bibitem{クリスティーズ古書ページ} \TeX 「紙の本を読むことで理解力向上」: \url{https://onlineonly.christies.com/s/livres-rares-et-manuscrits/lots/2054?filters=&page=2&searchphrase=&sortby=LotNumber&themes=}

    \bibitem{ITmedia} \TeX 「著作権法改正(令和3年5月26日 著作権法一部改正案が可決・成立)」: \url{https://www.itmedia.co.jp/news/articles/2106/24/news052.html}

    \bibitem{jstage} \TeX 「電子書籍利用者と紙書籍利用者の意識や行動の差異に関する
    研究 *」: \url{https://www.jstage.jst.go.jp/article/nig/52/1/52_061/_pdf}

\end{thebibliography}


\section{アンケート}

\subsection{Q1:調べる、討論する、考えるというプロセスはあなたにとってどのように有効でしたか。一 人で考えるのと比較して述べてください。}
討論の中で考えがまとまらずうまく伝えれないということがよくあり、思いついたことをすぐに人に伝えるのは難しいと感じたが、そのような場面は生活していると往々にしてあるので討論することは重要だと感じた。

\subsection{Q2:今回のようなレポートは何がよかったですか。何が大変でしたか。}
実際にテキストベースでものが完全な状態で残っているとこの人の意見はこうだったという事実が残せるがそれぞれのメモだと事実だったかという判断が難しいなと感じた。
テキストベースの議論は結構有用なのではと思った。
\subsection{Q3:リフレクション (今回の課題で分かったこと)・感想・要望をどうぞ。}
今回の課題について、ネットの記事とかの参照が多く実際の論文にあたったりすることができなかったのが残念だった。個人の見解が多くなってしまったのでこれからは事実の部分をしっかりと準備して意見しようと思った。
\end{document}
